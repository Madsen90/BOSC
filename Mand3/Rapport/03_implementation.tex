\section{Implementation}
I dette afsnit vil der gås lidt mere i dybden med hvordan de nævnte løsninger er implementeret (se afsnit \ref{sec:design}).

\subsection{Page fault handler}
Page handleren er den metode, der bliver kaldt, når der smides en page fault. Først bliver der checket, om page fault'en er en write request, hvilket ville gøre sideudskiftning unødvendig.

Sidenhen checkes der, om der er nogen fri frame, ved at gennemløbe vores frametable (map). Her chekkes der om der er nogen frame, der aldrig har været brugt; at den i frametable er -1.\footnote{Her kunne en optimering være ikke at gennemløbe map'et, efter alle frames er blevet tildelt første gang, eftersom de aldrig vil blive frie igen før programmets afslutning}

Hvis der ikke er nogen frame, der ikke er i brug, kommer frameSelecter i brug. Den bruger en af sideudskiftningsalgoritmerne til at finde den optimale frame at skifte ud. Hvis pagen der er mappet til det pågældende frame har skrive-rettigheder, bliver dataet på framet skrevet til disken, og den pågældende frame sættes til at være fri. Dette sker fordi vi antager at enhver page der har fået skriverettigheder også rent faktisk har skrevet til framet, og således er data blevet ændret ift. til hvad der ligger på disken. Derefter opdateres vores frametable og den frie frame mappes den page, der har fremprovokeret en page fault.

Derefter læses det ønskede data ind fra disken til den nu mappede frame.

\subsection{Sideudskifningsalgoritmer}
Der er implementeret tre sideskiftnings algoritmer. Deres implementation bliver kort beskrevet nedenfor.

\subsubsection*{Rand}
Rand algoritmen sætter bare pointeren til den frie frame til et vilkårligt tal fra 0-N, hvor N er tallet af frames. Funktionen srand48 bliver brugt til seeding og lrand48 bliver brugt til at generere tal.

\subsubsection*{FIFO}
FIFO algoritmen virker som en FIFO kø. Hvis ingen frame er tilgængelig og en frame skal skiftes ud, så skifter algoritmen den tidligst mappede frame ud med en ny, og flytter framen bagerst i køen.

Det gøres i praksis ved at opretholde datastrukturen FIFOData, der rent faktisk kun består af et heltal. Det heltal bliver inkrementeret hver gang der er brug for at loade nyt data ind på en frame indtil vi rammer tallet nframes, antallet af frames, så starter vi fra 0, den første side, igen ved hjælp af modulo operatoren.

Implementationen virker som beskrevet, under den antagelse at sider aldrig frigives og at frie frames uddelles i rækkefølge. Og da alle frie frames først bliver tildelt sekventielt og aldrig frigivet, må antagelsen holde vand.


\subsubsection*{Custom}
\label{subsubsec:custom}
Custom algoritmen er en version af en tilnærmet "Last recently used"-algoritme (LRU). For at kunne få et billede af hvornår en side sidst har læst eller skrevet til hukommelsen, så fjernes alle siders skriverettigheder med et fast interval (LRUTIME). Alle sider, der nu enten vil skrive eller læse fra hukommelsen, vil nu blive fanget i page\_fault\_handleren(). Her registreres det, at siden er blevet tilgået indenfor denne periode. For at undgå at disse fremprovokerede faults ikke ender i flere disk tilgange, holdes der styr på, om en pågældende side allerede er mappet til den fysiske hukomelse, og hvilke rettigheder den har haft tidligere.

Hver side i den virtuelle hukkomelse bliver tildelt to heltal, som bruges til at holde styr på, hvornår siden sidst er blevet tilgået, dens historie og hvilke rettigheder siden havde sidst, den havde adgang til den fysiske hukommelse. Historikken bliver brugt som beskrevet i "Operating System Concepts" på side 410. Hvis flere sider har samme laveste historik, så vælges den som er koblet til den laveste frame. Det betyder, at en fifo-struktur ved samme historik ikke er garanteret. Det kunne være løst ved at bruge en hægtet liste som historik.
