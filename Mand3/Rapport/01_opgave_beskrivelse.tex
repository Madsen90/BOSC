\section{Opgave beskrivelse}

I denne opgave bliver der implementeret en simpel virtuel hukommelse. Den virtuelle hukommelse er baseret på demand paging, og løsningen gør brug at allerede udlevert kode.

Den udleverede kode består af en virtuel side tabel, en virtuel disk og en ufulstændig main metode.

Main metoden tager fire parametre som input

\begin{itemize}
\item Antal virtuelle sider (pages)
\item Antal fysiske sider (frames)
\item Hvilken sideskiftning algoritme der skal bruges (rand|fifo|custom)
\item Hvilken simuleret programkørsel der skal udføres(focus|sort|scan)
\end{itemize} 

Vores opgaver er således at:

\begin{itemize}
\item Implementere en page fault handler, der skifter sider ind og ud, når sidetabellen er fuld.

\item Implementere tre sideskiftnings algoritmer: 

\begin{description}
\item[fifo] - Algoritmen skal virke som en FIFO kø, hvor den side der er kommet i brug først, er den første der skiftes ud. 
\item[rand] - Ikke en engentlig algoritme, da den bare skal vælge en vilkårlig side, der skal skiftes ud.
\item[custom] - Vores egne algoritme, der skal lave færre disk tilgange end de to andre algoritmer.
\end{description}

\item Ændre main metoden, så den parser tekststrengen fra kommanolinjen, så man kan vælge hvilken sideskiftnings algoritme programmet skal bruge.

\item Lave noget statestik over hvor mange disk tilgange hver algoritme laver, så vi rent faktisk ved, at vores custom algoritme er bedre end de andre to.
\end{itemize}